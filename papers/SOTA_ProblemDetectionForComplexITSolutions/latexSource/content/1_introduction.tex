Modern high demands on IT industry cannot be satisfied with a one-program single solution any more. Current software architectures rely on multiple part solutions of many other vendors which are combined to solve new problems. We think, it can be said, that most of software products you can find on the market, both open source as commercial, are really reliable with minimum run-time failures, otherwise they would lose their position and reputation. Plus the is a new common rend, that programs are becoming self-healing, so it a problem occurs, it finds a way how to reach a stable state on its own for instance as describes this Google patent \cite{khan2011self}. But developers can only focus on their own products, so if you are designing your service using these products which are relying on each other and also the effect of non-laboratory environment (network latency and unreliability, limited memory and machine resources, ...) can cause unpleasant troubles.

Every down time of a service comes along with consequences which can be in a form of a money penalization, losing of user preferences of data consistency problem. None of these is good for a company so every service owner has to have some people, DevOps engineers of System administrators, who are taking care of his service and who can immediately react if some failures are observed. Such person then have to investigate what is the root cause of an incident, most often by viewing related program log files and then has to d appropriate action to make system running correctly. You can be sure, that somebody had dealt with exactly the same of very similar problem, so it is possible to google a solution, but to do that, you need all related system and software information to identify the problem similarity. Such investigation is not an easy task, it is always hard to determinate which all components took a part at the incident.

In this paper we would like to identify the challenges of this problematic in Section \ref{sec:problemStatement}. The main focus is payed at to failure prediction and early detection and also to solution suggestion, so service maintenance can be maximally automated. Section \ref{sec:currentSolution} lists most important current frameworks and real implementations which are heaping to automate service operations, We are also trying to compare all mentioned solution and highlight current weak sides of them. Then we proposed our framework idea which take the best of all existing solution, combine them all together using our point of view on the problematic. Please keep in mind, this our solution has never been implemented yet and it is only a general idea.
